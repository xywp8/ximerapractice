\documentclass{ximera}
%% handout
%% space
%% newpage
%% numbers
%% nooutcomes

\documentclass{ximera}
\title{new activity}
\begin{document}
\begin{abstract}
An attempt to write a new activity.
\end{abstract}
\maketitle
\end{document} %% we can turn off input when making a master document

\outcome{Understand a first example of the Ximera style.}
\outcome{Have a nice basic example to work from.}

\title{Introduction to Real Numbers}


\begin{document}
\begin{abstract}
This section covers the notes of real numbers. 
\end{abstract}

\noindent {\LARGE \textbf{1.5 Algebraic Expressions:}}

{\LARGE \hspace{1in} \textbf{ Translating, Evaluating, and Simplifying}}

\noindent \rule{7.35in}{2pt}

\vspace{.2in}

\noindent \begin{tabular}{|c|}
\hline 
Definition\\
\hline 
\end{tabular} A \rule{1in}{.5pt} is a number, a variable, or the product or quotient of numbers and variables

$~$

\rule{1in}{.5pt} are seperated by $\displaystyle +$ signs.

$~$

$~~~~~~~~~~$ Any time there is a $\displaystyle -$ sign, it can be written with a plus sign as follows $$3x^2+2x-7=3x^2+2x+(-7)$$ 

$~~~~~~~~~~$ so the terms of this expression are $3x^2, ~ 2x,$ and $-7$.

$~$

$~$



\noindent Ex 1)  Translate each algebraic expression to words.

$~$

\begin{tabular}{l|c}
Algebraic Expression & $~~~~~~~~~~~~~~~$Translation$~~~~~~~~~~~~~~~$\\
\hline
$~$\\
\textbf{a.} $~~~~\frac{1}{5}x$ & $~$\\
$~$\\
\hline

$~$\\
\textbf{b.} $~~~~5-x$ & $~$\\
$~$\\
\hline

$~$\\
\textbf{c.} $~~~~s\div(-4)$ & $~$\\
$~$\\
\hline

$~$\\
\textbf{d.} $~~~~n+(-10)$ & $~$\\
$~$\\
\hline

$~$\\
\textbf{e.} $~~~~\frac{5}{8}m$ & $~$\\
$~$\\
\hline
$~$\\
\textbf{f.} $~~~~6x-7$ & $~$\\
$~$\\
\hline
$~$\\

\textbf{g.} $~~~~10(a+b)$ & $~$\\
$~$\\
\hline
$~$\\

\textbf{h.} $~~~~\frac{3}{p-q}$ & $~$\\
$~$\\
\end{tabular}



\pagebreak

\noindent Ex 2) Translate each phrase to an algebraic expression.

$~$

\begin{tabular}{l|c}
$~~~~~~~~~~~~~~~~~~~~~~$Phrase & $~~~~~~~~~~$Translation$~~~~~~~~~~~~~~~$\\
\hline
$~$\\
\textbf{a.} $~~~~\frac{1}{5}$ of a number& $~$\\
$~$\\
\hline

$~$\\
\textbf{b.} The sum of a number and negative 1 & $~$\\
$~$\\
\hline

$~$\\
\textbf{c.} The difference between $x$ and negative 2 & $~$\\
$~$\\
\hline

$~$\\
\textbf{d.} The ratio of 4 and $n$ & $~$\\
$~$\\
\hline

$~$\\
\textbf{e.} The product of negative 3 and d & $~$\\
$~$\\
\hline
$~$\\
\textbf{f.} 12 less than the product of 3 and $y$ & $~$\\
$~$\\
\hline
$~$\\

\textbf{g.} The quantity of $a$ plus $b$ divided by the quantity $a$ minus $b$ & $~$\\
$~$\\
\end{tabular}

\vspace{.25in}

\begin{center}
\begin{tabular}{|l|}
\hline
\textbf{To Evaluate an Algebraic Expression}  $~~~~~~~~~~~~~~~~~~~~~~~$\\
\hline
$\bullet$\\
$~$\\
\hline
\end{tabular}
\end{center} 

$~$

\noindent Ex 3)  Find the value of each expression for $a=4$, $b=-1$, $c=-2$, and $d=3$.

$~$

\textbf{a.}$~~~~$$\displaystyle 5a-1$ \hspace{2.5in} \textbf{b.}$~~~~$$\displaystyle -c^4$

\vspace{.75in}

\textbf{c.}$~~~~$$\displaystyle (-c)^4$ \hspace{2.5in} \textbf{d.}$~~~~$$\displaystyle 3b-2d$

\vspace{.75in}

\noindent Ex 4)  Find the value of each expression when $x=-2$, $y=-1$, and $z=3$.

$~$.

\textbf{a.}$~~~~$$\displaystyle \frac{x+z}{x-y}$  \hspace{2.5in} \textbf{b.}$~~~~$$\displaystyle \frac{x-3z}{y}$

\vspace{.75in}

\textbf{c.}$~~~$$\displaystyle -4z^2-4(y-z)$ \hspace{2in} \textbf{d.}$~~~~$$\displaystyle 5y^2+z^3$

\pagebreak

\noindent \begin{tabular}{|c|}
\hline 
Definition\\
\hline 
\end{tabular} \rule{2in}{.5pt} are terms that have the \textit{same variables} with the 

\textit{same exponents}.

$~$

$~$

\noindent Ex 5) Combine like terms.

$~$

\textbf{a.}$~~~~$$\displaystyle -7y-y$ \hspace{2.5in} \textbf{b.}$~~~~$$\displaystyle a-4a+8b$

\vspace{.75in}

\noindent Ex 6) Simplify, if possible.

$~$

\textbf{a.}$~~~~$$\displaystyle y^2-6y$ \hspace{1.3in} \textbf{b.}$~~~~$$\displaystyle 2n^2-5n^3+7n^3$ \hspace{1.3in} \textbf{c.}$~~~~$$\displaystyle9xy^2-xy^2$

\vspace{2in}

\textbf{d.}$~~~~$$\displaystyle 5\left(y-\frac{2}{5}\right)+8$ \hspace{2.5in} \textbf{e.}$~~~~$$\displaystyle -(4a-9b)$

\vspace{2in}

\textbf{f.}$~~~~$$\displaystyle 3y+7-(2y-5)$ \hspace{2.5in} \textbf{g.}$~~~~$$\displaystyle 5(5y+8)-3(7y+10)$

\vspace{2in}

\textbf{h.}$~~~~$$\displaystyle 13-2[-15y+4(3y-1)]$



\end{document}
